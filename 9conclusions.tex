\chapter{Conclusions}
\label{chapter:conclusions}

This thesis set out to implement and trial a system for serving students a means to execute their program code in a browser with an interactive and modern GUI that use media content. EDCAT offered the sufficient functionality to validate the idea and prove its worth in the programming education context. The system was easy to use for students, the teaching assistant as well as the instructor and it did not require extra effort compared with the use of traditional and a fully local environment. The distributed design of the system proved useful and the chosen technologies supported the needs of the course.

EDCAT allowed to hide potentially difficult code from the students' eyes and produce modern and attractive GUIs. The students needed to download very little code and the updates to the exercise materials reached every student as soon as published. However, the debugging functionalities were not comparable to the locally executed GUIs as the error messages were not properly shown, not to mention the lack of a debugger available in an IDE.

The results indicate that the technical setup was defective in the first trial run of the system and proper server resources should be allocated for the system to handle the huge number of compilations within an appropriate time limit. Furthermore, technical details, in terms of browser dependencies and other limitations, must be thoroughly investigated in order to avoid the problems arisen in the second assignment round.

The feedback given both by students and the teaching body has already proven useful and must be utilised when the system should be further developed. The support for comprehensive error output should be added by the use of a simple console. This requires modifying the GUI code as well as taking use of the source maps produced by Scala.js compilation. Additionally, the feedback revealed that a more straightforward submission procedure should be implemented in the future to prevent potential confusions caused by two step submission used in the trial. Furthermore, guidance on using the environment should be given so that students could utilise its full potential.

There is still much work in developing the functionality to aid the work of teaching assistants who could grade the submissions and tutor students with the better support from the system. Implementing test suites would require little effort and equip the teaching assistant with a uniform assessment basis. The feedback functionality should be modified towards a bidirectional implementation and used for providing help also during the completion of the assignments. Additionally, functionality to gather data on student's process should be developed to allow following students' problem solving process.

Based on the results we achieved with the trial run of EDCAT, we can recommend the use of a similar system for GUI implementation and distribution. EDCAT has potential showcasing program code written by students used in an appealing GUI built with modern technology. The system is easy for the teacher to manage and composing good GUIs with the use of Scala.js takes significantly less effort than using Swing.


\section{Future Work}
\label{section:futureWork}

In this first version only the bare minimum functionality for supporting the teaching assistants' work was implemented. As already discussed, developing tools for remote tutoring and especially aiding the grading process are areas worth looking into. The remote tutoring is somewhat supported but requires an additional environment for discussion between course staff and students. There is no sense in fully integrating a discussion platform with EDCAT but adding links to starting a new question regarding the submission could lower the barrier to using the remote help. Better support for online tutoring is crucial if EDCAT should ever be used in a course with a big student body, similar to \emph{Programming 1} course.

Grading students' work fairly and efficiently required downloading students' code in order to run it locally against tests. The testing system could be integrated into EDCAT so that assistants could devise a test suite together online and use it with every submission. This would also allow the partial --- or if desired the full --- automated grading of students' submissions and teaching assistants could concentrate on grading style and clarity, and writing feedback. Some parts of the same system could also be open for students, so that they could easily define their own tests.

An important skill for any programmer is debugging --- a skill that novices need and are lacking in the most. Supporting the development of debugging skills could be supported by the trial GUIs. Should an error occur, students would be given not only the error message but also some aid as to what might cause the error on the particular line and how one should proceed in order to correct it.

Forcing the trial runs online allows for an unprecedented vantage point for monitoring students' efforts solving assignments they are given. The unfinished submissions might provide some interesting perspective into how students go about solving the assignments and the problems that might occur during the process. EDCAT might provide the means to study students' problem solving unobtrusively.

From the theoretical point of view, it would be interesting to see how well our course materials succeed in teaching the students about media programming. The exercises are in need of formal evaluation and possible redesign. Similarly, it would definitely be interesting to conduct a longitudinal study with our students, to evaluate how they choose their minor and what affects their decision making. From the technical point of view, the potential of using Scala.js in programming education should be further explored in the future. Additionally, the exercise design should be further developed to allow students to extend their solution with more freedom.
